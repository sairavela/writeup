% Conclusion
% * Describe each problem, research design, and findings (ans to prob)
% * Recommendations
\chapter{Conclusion}

We have implemented an additional two extensions to Sloop to improve the
classification performance.

\section{Relevance Feedback}
A relevant feedback simulator and the crowdsourced
relevance feedback integration for Sloop (Sloop MTurk) have been implemented to
study the impact of relevance feedback on the ranked retrieval data, as well as
improve the correctness of the result sets. The experiments have confirmed that
relevance feedback dramatically accretes precision and recall, despite the
presence of errors. We demonstrated that certain sampling methods outperform others in terms
of the cost. Exploiting the accumulation of retrieved information, we
introduced a new sampling method that takes advantage of the difference between
the CDF of score, given matches/non-matches. We then compared the performance of
our adaptive sampling approach to the other static approaches.

\section{Convolutional Neural Network}

We presented a new architecture for a learning similarity metric in a large
dataset and experimented with several algorithms to find the solution best
suited for our animal identification scheme. 

We proposed a new architecture that comprises a feature extractor and a match
recognizer. The feature extractor is implemented by a pre-trained convolutional
network (AlexNet) with the output layer removed. Given a pair of two images,
the feature extractor outputs two fixed-size feature vectors. We then take the
absolute difference of the feature vector and pass the resulting vector into a
support vector machine classifier, which has been proven to be the most robust and
suitable model for our application. Finally, the classifier outputs whether the
two images contain the same individual animal. According to the results, it
is able to achieve very high precision with good recall while being completely
automated and independent from human involvement. 

