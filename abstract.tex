% $Log: abstract.tex,v $
% Revision 1.1  93/05/14  14:56:25  starflt
% Initial revision
%
% Revision 1.1  90/05/04  10:41:01  lwvanels
% Initial revision
%
%
%% The text of your abstract and nothing else (other than comments) goes here.
%% It will be single-spaced and the rest of the text that is supposed to go on
%% the abstract page will be generated by the abstractpage environment.  This
%% file should be \input (not \include 'd) from cover.tex.


% In this thesis, I designed and implemented a compiler which performs
% optimizations that reduce the number of low-level floating point operations
% necessary for a specific task; this involves the optimization of chains of
% floating point operations as well as the implementation of a ``fixed'' point
% data type that allows some floating point operations to simulated with
% integer arithmetic.  The source language of the compiler is a subset of C,
% and the destination language is assembly languagesadsdasdsd for a
% micro-floating point CPU.  An instruction-level simulator of the CPU was
% written to allow testing of the code.  A series of test pieces of codes was
% compiled, both with and without optimization, to determine how effective
% these optimizations were.

The ability to identify individual animals is crucial for non-invasive
ecological monitoring and conservation planning. This project proposed two
improvements to the recognition process and ranked retrival of Sloop, the first
image retrieval engine that couples automated pattern recognition with
crowd-sourced relevance feedback for individual animal identification.

With a crowd-sourc`ed relevance feedback simulator, we report a number of
studies corroborating the acceration of precision and recall of the retreival
results after various rounds of relevance feedback and the effects of the error
propagation.

Then, we describe Sloop MTurk, the crowdsourced relevance feedback integration
of Sloop.

In the later part, we propose a new architecture for animal pattern
recognition, which could possibly reduce the system necessity for human
involvement in feature extraction using transfer learning. Then, we experiment
with a variety of binary classifiers in order to identify the algorithm that
accomplish good performance on our data. Our results reveal that Convolutional
network with linear support vector machine with radial basis kernel function
(SVM-RBF) achieves a very robust performance on Otago and Grand data.


