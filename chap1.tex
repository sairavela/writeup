% Introduction (context in terms of content of the project)
% * Significance (who will benefit? contribution of the study)
% * Statement of the problem (problem must be reflected to the title)
% * Conceptual framework (problems in relation to relevant literatures,
% summarize the major variables -- independent variables/cause,
% dependent/effect, other influential vars)
%   - Existing research and its relevancy
%   - Key idea of my approach
%   - discuss variables related to the problem
%   - Conceptualize relationship between variables
% * Scope and delimitation
% (* list out technical terms)

\chapter{Introduction}

Unbiased information about animal population ecology gives biologists vital
information about the effects of different physical or biological factors on
the distribution and abundance of animal species, which plays a key role in the
development of effective conservation strategies for rare and endangered
species. Typical approaches that biologists take to estimate the size of a
population of a given species often involve physically marking the animals.
These approaches are not only expensive both in terms of time and cost, but
they are also invasive for the animals.

The ability to identify an individual animal by recognizing it in photographs
provides a non-invasive alternative that allows researchers to monitor the
species' diversity and dispersal.  Researchers can track movements and observe
the genetic variation of a species by comparing each member's images with the
all the existing images collected at different times and locations. However,
the arduous task of comparing over a thousand images of every individual animal
and their potential matches makes manual reviews impractical for large
collections. To alleviate the problem, we need to automate the recognition
process using computer-based image recognition techniques.

The considerable progress we have seen in computer vision is largely due to
local descriptor-based algorithms, such as SIFT~\cite{lowe04}, and
SURF~\cite{surf08}, etc. The field of computer vision covers a wide range of
topics. For the purpose of animal image biometrics, this thesis will deal
mostly with the image classification and recognition problems. Despite the
stellar performance of these algorithms in the past decades, there exists a
substantial gap between algorithmic and human-level performance. In this work,
we present two improvements, to \emph{Sloop}, the first pattern retrieval
engine for animal biometrics incorporating crowd-sourced relevance feedback.
These improvements are shown to bridge the performance gap between these
machine-based algorithms and humans.

In the last few years, deep convolutional neural networks~\cite{lecun95,
kriz12} have outperformed SIFT and other descriptor-based techniques by a large
margin in both object recognition and classification tasks~\cite{kriz12,
fisher14, ILSVRC15}. In fact, the architecture has demonstrated recognition
accuracy comparable to humans in several visual recognition tasks, such as
recognizing faces~\cite{deepface14} and handwritten digits~\cite{mnist13}.
Motivated by the preceding achievements, we integrate a pre-trained
convolutional neural network into a new python version of Sloop, \emph{SlooPy},
as a separate image processing workflow.

Despite the scalability and advantages of computational speed in automatic
pattern recognition, some classification errors may occur and rapidly
propagate. Some degree of human involvement may benefit the identification
process. Not only can user input resolve the errors, it can also be used to
train a machine learning model. The model can incrementally learn from the
accumulation of user-input data in our retrieval system. This creates a
positive feedback loop where the model learns its mistakes from the previous
iterations and re-evaluates its strategy based on the gold standard responses
from a human given at the previous iteration. However, this is out of scope
for our work. In this project, we focus on the problem of how we can maximize
the information gain from a given amount of user input.

This project involves design and implementation of two additional features of
Sloop: relevance feedback integration and convolutional neural network
integration.

\section{Problem Statements}

\subsection{Relevance Feedback} % (fold)
\label{sub:relevance_feedback}
One problem of interest is assigning tasks to workers with the goal of maximizing
the quality of completed tasks at a low prices or subject to budget
constraints.
% subsection relevance_feedback (end)

\subsection{Image Recognition} % (fold)
\label{sub:image_recognition}
The emergence of convolutional neural networks pushes forward the frontiers of all
domains of computer vision~\cite{lecun95}. Recent studies show that
convolutional neural network architectures clearly dominate both handcrafted
features and traditional orientation-based local descriptors, such as
SIFT~\cite{lowe04}, and SURF~\cite{surf08}, etc.\ in classification
tasks~\cite{fisher14,kriz12,prelu15,ILSVRC15}.

The second part of this thesis presents a new architecture whose goal is to
improve the identification accuracy as well as curtail or eliminate human
involvement. We compare the recognition ability of the two algorithms: SIFT and
convolutional neural networks.
% subsection image_recognition (end)

\section{Challenges}

The animal identification task involves two major challenges: the \emph{image
feature extraction problem}, and the \emph{pattern recognition problem}.

The image feature extraction problem involves locating the animal in the image,
and extracting the features required for matching. The choice of feature
selection varies from species to species. The extracted feature object of an
image is then passed into the pattern recognition algorithm to find the matches
that exist in the system. The current version of Sloop locates the animal and
necessary features in an image by having the user click on fiducial points in
the image, and then calculates the feature vector using SIFT\@. It performs the
matching by scoring the similarity of the SIFT objects between images.

Traditional approaches in machine learning generally require training samples
to be available for all categories. Moreover, such approaches are designed to
handle only the dataset with a finite, preferably small, number of categories.
Nevertheless, our application requires the ability to recognize high
dimensional input, whose categories are not known in advance. In addition,
while the number of categories can be very large, the number of examples per
category can be very small.
