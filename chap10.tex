% Conclusion
% * Describe each problem, research design, and findings (ans to prob)
% * Recommendations
\chapter{Future Work} % (fold)
\label{cha:future_work}

\section{Convolutional Neural Network}

One possible improvement is to tailor and train the parameters of the
Convolutional Network to best fit our data. However, this requires a much
larger dataset of labeled images. Therefore, we need to increase the size of
our dataset.

In order to properly evaluate the new system performance, we should use the
SIFT algorithm in the previous system as our baseline. We have already
generated the SIFT object representing each images available in the system
however, they are generated from a sets of keypoints that are manually marked
so that it is able to acheives 99.99\% ranking accuracy. With such biase, it is
hard to evaluate the relative performance between the new and the old system.

Further, we can evaluate the performance of the model by analyzing the receiver
operating characteristics (ROC), or ROC curves, which illustrates the
performance of a classifier model. ROC exhibits the relationship between true
positive rate (TPR) and the false positive rate (FPR), which provides tools to
select possibly optimal models and to discard suboptimal ones.

Since the datasets contain the human input, the manual input data can be
utilized to improve the identification result as a one of the features. The
definition of number of human interaction depends on the action required to
complete a specific classification task. For instance, if the classification
task requires user to click on the photos, the number of user clicks will then
determine the number of human interaction. The classification results of the
completely automated feature extractor and semi-automated feature extractor
should be compared to determine the optimal human involvement.

% chapter future_work (end)